\documentclass{book}
\usepackage{bm}
\usepackage{graphicx}
\usepackage{amsmath}
\usepackage{amssymb}
\usepackage{physics}
\usepackage{braket}
\usepackage{algorithmicx}
\usepackage{algorithm, algpseudocode, geometry}
\usepackage{hyperref}
\usepackage{booktabs}
\usepackage{caption}
\title{Thesis}
\date{}
\begin{document}
\maketitle
\tableofcontents

\chapter{Nuclear structure}
\section{NN interaction}
\subsection{Experimental evidence}
\subsection{Physical properties}
\subsection{Meson exchange}
\section{Nuclear phenomenology}
\subsection{Nuclear density}
\subsection{Nuclear radius}
\subsection{Deformed nuclei}
\subsection{Pairing}
\section{Mean field approach}
\subsection{Woods-Saxon potential}
\subsection{Nuclear Spin-Orbit interaction}
\subsection{Shell structure}

\chapter{Hartree-Fock Theory}
\section{Preliminaries}
\subsection{Variational principle}
\subsection{Slater determinant}

\section{Formal theory}
\section{Hartree-Fock method}

\chapter{Numerical methods}
\section{Finite differences}
\section{Numerical mesh}
\section{Differential operators}

\chapter{General Conjugate Gradient}
\section{Conjugate gradient}
\section{Non-linear conjugate gradient}
\section{General conjugate gradient}

\chapter{Skyrme interaction}
\section{Energy functional}
\section{Coulomb field}
\section{Results for Oxygen-16}

\chapter{Results on deformed nuclei}

\end{document}