\chapter{Energy functional}
Now that the theoretical and numerical framework is clear, we can investigate a plausible nucleonic interaction, which in the present work, takes the form of the Skyrme interaction.
\\It was first proposed by Tony Skyrme in 1958 \cite{SKYRME1958615} as a zero range force between nucleons, and has been used successfully as the building block of nuclear structure.
\\Nowadays, the standard form is slightly enriched to be more general \cite{CHABANAT1997710}. It comprises a two-body interaction, which reads
\begin{align}
v^{(2)}(\mathbf{r}_1, \mathbf{r}_2) &= t_0 \left(1 + x_0 P_\sigma \right) \delta(\mathbf{r}) \\
&\quad + \frac{1}{2} t_1 \left(1 + x_1 P_\sigma \right) \left[ \mathbf{P}'^2 \delta(\mathbf{r}) + \delta(\mathbf{r}) \mathbf{P}^2 \right] \\
&\quad + t_2 \left(1 + x_2 P_\sigma \right) \mathbf{P}' \cdot \delta(\mathbf{r}) \mathbf{P} \\
&\quad + \frac{1}{6} t_3 \left(1 + x_3 P_\sigma \right) \left[ \rho(\mathbf{R}) \right]^\sigma \delta(\mathbf{r}) \\
&\quad + i W_0 \boldsymbol{\sigma}\cdot \left[ \mathbf{P}' \times \delta(\mathbf{r}) \mathbf{P} \right]
\end{align}
And a three body interaction, that is
\begin{equation}
v^{(3)}(\mathbf r_1, \mathbf r_2)=\frac 1 6 t_3 \left(1 + x_3 P_\sigma \right) \left[ \rho(\mathbf{R}) \right]^\sigma \delta(\mathbf{r}) 
\end{equation}
Where 
\begin{align*}
\\\mathbf{r} &= \mathbf{r}_1 - \mathbf{r}_2
\\\mathbf{R} &= \frac{\mathbf{r}_1+\mathbf{r}_2}{2}
\\\mathbf{P} &= \frac{-i(\nabla_1 - \nabla_2)}{2}
\\\boldsymbol{\sigma} &= \boldsymbol{\sigma}_1 + \boldsymbol{\sigma}_2
\\\mathbf{P}_\sigma &= \frac{(1+\boldsymbol{\sigma}_1\cdot\boldsymbol{\sigma}_2)}{2}
\end{align*}
Primed operators refer to the complex conjugate acting on the bra space.
\\This formulation respects all symmetries required of a non relativistic nuclear interaction (Galilean boost, particle exchange, translation, rotation, parity, time reversal and translation).
\\Taking the expectation value of the many body hamiltonian, in the Hilbert space of Slater determinants, yields
\begin{equation}
    \expval{H} = \bra{\Psi} H \ket{\Psi} = \int \mathcal H (\mathbf r)d\mathbf r = \mathcal E_\text{Skyrme}
\end{equation}
In the case of even-even nuclei, time-odd components of the functional reduce to zero, leaving \cite{stevenson2019low}
\begin{equation}
    \mathcal E_\text{Skyrme} = \int \sum_{t=0,1}\bigg\{C_t^\rho [\rho_0]\rho_t^2+C_t^{\Delta \rho}\rho_t\nabla^2\rho_t+C_t^{\nabla\cdot J}\rho_t\nabla\cdot \mathbf J_t + C_t^\tau\rho_t\tau_t\bigg\}d\mathbf r
\end{equation}
Here, $t=0,1$ refers to the isoscalar and isovector components of the densities, e.g.
\begin{align*}
    \rho_0 = \rho_p - \rho_n
    \\\rho_1 = \rho_p + \rho_n
\end{align*}
Where
\begin{align}
    C_0^\rho &= +\frac 3 8 t_0 + \frac 3 {48} t_3\rho_0^\sigma 
    \\C_1^\rho &= -\frac 1 8 t_0(1+2x_0)- \frac 1 {48} t_3(1+x_3)\rho_0^\sigma 
    \\C_0^\tau &= +\frac 3 {16} t_1 + \frac 1 {16} t_2 (5+4x_2)
    \\C_1^\tau &= -\frac 1 {16} t_1(1+2x_1)+\frac 1 {16}t_2(1+2x_2)
    \\C_0^{\Delta \rho} &= -\frac 9 {64}t_1+\frac 1 {64}t_2(5+4x_2)
    \\C_1^{\Delta \rho} &= +\frac 3 {64}t_1(1+2x_1)+\frac 1 {64}t_2(1+2x_2)
    \\C_0^{\nabla\cdot J} &= -\frac 3 4 W_0
    \\C_1^{\nabla\cdot J} &= -\frac 1 4 W_0
\end{align}
As outlined in previous chapters (REF), we can now derive the Kohn-Sham equations, by constraining orthonormality and enforcing the variation of the functional to be zero. What we end up with is
\begin{equation}
    \bigg[-\nabla\bigg(\frac{\hbar^2}{2m^{*}_q(\mathbf r)}\nabla \bigg) + U_q(\mathbf r) + \delta_{\text{q,proton}}U_C(\mathbf r)-i\mathbf B_q(\mathbf r)\cdot(\nabla \times \boldsymbol\sigma) \bigg]\varphi_\alpha=\varepsilon_\alpha\varphi_\alpha
\end{equation}
The index $q=n,p$ refers respectively to the neutron and proton quantites.
\\Each term is here detailed
\begin{align}
    \frac{\hbar^2}{2m^{*}_q(\mathbf r)} &= \frac{\hbar^2}{2m}+\fdv{\mathcal H}{\tau_q}
    \\U_q(\mathbf r) &= \fdv{\mathcal H}{\rho_q}
    \\\mathbf B_q(\mathbf r) &= \fdv{\mathcal H}{\boldsymbol{\mathbf J_q}}
\end{align}
The coulomb field $U_C$, which is present only in the single particle equation for protons, will be properly developed in section (REF).
\\Following the rules for functional derivatives, outlined in the appendix (REF) for our particular case, we have
\begin{align}
    \frac{\hbar^2}{2m_q^*(\mathbf r)} =& +\frac{\hbar^2}{2m} \\&+ \frac 1 8 [t_1(2+x_1)+t_2(2+x_2)]\rho(\mathbf r) \\&- \frac 1 8 [t_1(1+2x_1)+t_2(1+2x_2)]\rho_q(\mathbf r ) \\\\
    U_q(\mathbf r) =& +\frac 1 8 [t_1(2+x_1)+t_2(2+x_2)]\rho \\&+ \frac 1 8 [t_2(1+2x_2)-t_1(1+2x_1)]\rho_q \\
    &+ \frac 1 8 [t_1(2+x_1)+t_2(2+x_2)]\tau \\&+ \frac 1 8 [t_2(1+2x_2)-t_1(1+2x_1)]\tau_q \\
    &+ \frac 1 {16} [t_2(2+x_2)-3t_1(2+x_1)] \nabla^2 \rho \\&+ \frac 1 {16} [3t_1(2x_1+1)+t_2(2x_2+1)] \nabla^2 \rho_q \\\\
    \mathbf W_q (\mathbf r ) = &+\frac 1 2 W_0 [\nabla\rho + \nabla \rho_q] \\&-\frac 1 8 (t_1 x_1 + t_2 x_2) \mathbf J + \frac 1 8 (t_1 - t_2) \mathbf J_q 
\end{align}
For ease of notation and implementation, unindexed densities refer to isovector quantites.






