\subsection{Shell model}
\subsubsection{Coulomb interaction}
Protons are positively charged, which means that beside the strong force we have to take into account the Coulomb interaction.
\\Keeping in theme with a mean field approach, we can approximate the Coulomb potential generated by the protons as the one of a uniform sphere, of charge $Z$, with radius $R$.
\begin{equation}
    v_{\text{C}}(r) = \frac{Ze^2}{4\pi\varepsilon_0} 
    \begin{cases}
        \frac{3-(r/R)^2}{2R} & r \le R \\
        \frac 1 r & r > R
    \end{cases}
\end{equation}
\subsection{Shell model calculation}
The final Hamiltonian for our mean field phenomenological nuclear model will be
\begin{equation}
    H_p = H_\text{WS} + H_\text{SO} + H_\text{C}
\end{equation}
for protons, while
\begin{equation}
    H_n = H_\text{WS} + H_\text{SO} 
\end{equation}
for neutrons.
\\Applying the theory explained so far, we get the following results



