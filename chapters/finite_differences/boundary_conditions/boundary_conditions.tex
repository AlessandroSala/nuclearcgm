
\subsubsection{Boundary conditions}
For a short range potential, we can expect bound solutions to decay rapidly near the boundaries, prompting for Dirichlet boundary conditions.
\\In a simple 1D case, where the domain is $[a, b]$, $\psi_0 = \psi(a)$, $\psi_{N-1} = \psi(b)$, the approximation for the derivatives asks for $\psi_{-1}$ and $\psi_N$.
\\Since we are not contemplating these points in the domain, we can simply set them to zero, which formally corresponds to solving
\begin{equation}
    \begin{pmatrix}
        0 & 0 & 0 
        \\0 & A & 0
        \\0 & 0 & 0
    \end{pmatrix}
    \begin{pmatrix}
        0\\\mathbf{\psi}\\0
    \end{pmatrix}
    = E 
    \begin{pmatrix}
        0\\\mathbf{\psi}\\0
    \end{pmatrix}
\end{equation}
This implies that the solution $\psi$ satisfies the boundary conditions for 
\begin{equation}
    \psi(a-h)=0 \text{ and } \psi(b+h)=0.
\end{equation}
\\Working with a sufficiently fine mesh, this slight deviation is physically negligible, since $h \approx 0$. Nonetheless, it's possible to satisfy the exact BC by working in the domain $[a+h, b-h]$. 
