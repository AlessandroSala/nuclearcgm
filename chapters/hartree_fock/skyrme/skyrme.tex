
\subsection{Skyrme interaction}
In nuclear physics, the interaction between two nucleons has a very complex functional form. This requires choosing an interaction potential which enables us to simplify the equations and solve the problem.
\\One of such potentials is the Skyrme interaction, whose formulation is based on the physical principle of a short range interaction among nucleons, consequence of the exchange of massive bosons as the force mediators.
\\It is made up of two parts, a two body interaction and a three body interaction.
\[V=\sum_{i<j}v_{ij}^{(2)}+\sum_{i<j<k}v_{ijk}^{(3)}\]
The standard modern parametrization for $v^{(2)}$ is (Chabanat 1998)
\begin{align*}
v^{(2)}(\mathbf{r}_1, \mathbf{r}_2) &= t_0 \left(1 + x_0 P_\sigma \right) \delta(\mathbf{r}) \\
&\quad + \frac{1}{2} t_1 \left(1 + x_1 P_\sigma \right) \left[ \mathbf{P}'^2 \delta(\mathbf{r}) + \delta(\mathbf{r}) \mathbf{P}^2 \right] \\
&\quad + t_2 \left(1 + x_2 P_\sigma \right) \mathbf{P}' \cdot \delta(\mathbf{r}) \mathbf{P} \\
&\quad + \frac{1}{6} t_3 \left(1 + x_3 P_\sigma \right) \left[ \rho(\mathbf{R}) \right]^\sigma \delta(\mathbf{r}) \\
&\quad + i W_0 \boldsymbol{\sigma}\cdot \left[ \mathbf{P}' \times \delta(\mathbf{r}) \mathbf{P} \right]
\end{align*}
Where 
\[\mathbf{r} = \mathbf{r}_1 - \mathbf{r}_2\]
\[\mathbf{R} = \frac{\mathbf{r}_1+\mathbf{r}_2}{2}\]
\[\mathbf{P} = -i(\nabla_1 - \nabla_2)/2\]
\[\boldsymbol{\sigma} = \boldsymbol{\sigma}_1 + \boldsymbol{\sigma}_2\]
\[\mathbf{P}_\sigma = (1+\boldsymbol{\sigma}_1\cdot\boldsymbol{\sigma}_2)/2\]
Primed operators refer to the complex conjugate acting on the left.
\\This formulation respects all symmetries required of a non relativistic nuclear interaction (Galilean boost, particle exchange, translation, rotation, parity, time inversion and translation).
\subsubsection{Three body interaction}
The three body term of the Skyrme force is encapsulated by the term 
\[\frac 1 6 t_3 \left(1 + x_3 P_\sigma \right) \left[ \rho(\mathbf{R}) \right]^\sigma \delta(\mathbf{r}) \]
Here, $\sigma$ in the exponent is a free parameter of the force.
\subsubsection{Energy functional}
Evaluating the energy functional \ref{eq:energy_functional}, we get
\begin{align}
    \bra{\Psi} H \ket{\Psi} = \int H(\mathbf{r}) d^3 r 
    \\\mathcal H(\mathbf{r}) = \mathcal{K} + \mathcal{H}_0 + \mathcal{H}_3 + \mathcal{H}_\text{eff}+\mathcal{H}_\text{fin}+\mathcal{H}_\text{so} + \mathcal{H}_\text{sg} + \mathcal{H}_\text{coul}
\end{align}
$\mathcal H$ depends on $\mathbf r$ through
\[\rho_q(\mathbf r) = \sum_{i} \abs{\phi^q_{i,s}(\mathbf r)}^2\]
\[\tau_q(\mathbf r) = \sum_{i} \abs{\nabla\phi^q_{i,s}(\mathbf r)}^2\]
\[J(\mathbf r) = \ldots\]
Where $i$ goes through all single particle states.
\\Taking the variation of $E[\Psi]$ with respect to $\phi_i^*$ we get a single particle equation
\begin{equation}
   \bigg(-\nabla \frac{\hbar^2}{2m^*(\mathbf r)} \nabla + U_q(\mathbf r)+\delta_{q, \text{proton}}V_c(\mathbf r)+\ldots\bigg)\phi_i = \varepsilon_i\phi_i 
\end{equation}
We will now see how to properly treat each term of the equation.
\subsubsection{Kinetic term}
Regarding the kinetic component, we end up with an effective mass, such that 
\[\frac{\hbar^2}{2m^*(\mathbf r)} = \frac{\hbar^2}{2m} + \frac 1 8 [t_1(2+x_1)+t_2(2+x_2)]\rho(\mathbf r) - \frac 1 8 [t_1(1+2x_1)+t_2(1+2x_2)]\rho_q(\mathbf r ) = \mu(\mathbf r)\]
This allows us to write
\begin{align}
    \nabla\bigg( \frac{\hbar^2}{2m^*(\mathbf r)} \nabla \phi\bigg)&= \nabla \mu(\mathbf r)\cdot \nabla \phi +\mu(\mathbf r)\nabla^2 \phi 
\end{align}
Where both $\nabla, \nabla^2$ can be readily evaluated in the previously illustrated finite difference scheme as matrix coefficients.
\subsubsection{Coulomb interaction}
By the Slater approximation, the Coulomb interaction becomes
\begin{equation}
    V_c(\mathbf r) = \frac{e^2}{2}\int \frac{\rho_p(\mathbf r')}{|\mathbf r-\mathbf r'|} d^3 r' - \frac{e^2}{2}\bigg(\frac {3}{\pi}\bigg)^{1/3}\rho_p^{1/3}\rho(\mathbf r)
\end{equation}
Looking at the integral, we can make some considerations regarding the discretization of the problem.
\\As a rough approximation, having $\rho_p$ evaluated on a grid $(i, j, k)$, the integral becomes
\[\int (.)d^3 r' \to \sum_{i'j'k'} (.) h^3\]
When $(i, j, k)\neq (i', j', k')$, the integrand can be easily evaluated, while the singularity in $\mathbf r = \mathbf r'$ needs further treatment.
\\The integral can be separated into 
\[\sum_{(i, j, k)\neq (i', j', k')} \frac{\rho_p(\mathbf r')}{|\mathbf r-\mathbf r'|} h^3 + V_\text{self} \]
Where $V_\text{self}$ will be the integral evaluated in the cell centered on the singularity.
\[V_\text{self} = \int_{\text{cell}}\frac{\rho_p(\mathbf r')}{|\mathbf r-\mathbf r'|} = \rho_p(i, j, k)\iiint_{\text{cell}}\frac{dx'dy'dz'}{\sqrt{x'^2+y'^2+z'^2}}\]
If the cell is cubic, the latter integral is known, yielding
\[V_\text{self} = 1.93928\cdot\rho_p(i, j, k)\cdot h^2\]


