\section{The Hartree-Fock method}
A many body system, like the nucleus, is made of indistinguishible particles from the standpoint of quantum mechanics.
\\Suppose to have a nucleus, with $A$ nucleons, where the mass difference between neutrons and protons is neglected.
\\The state of the system will be described by a wavefunction $\Psi(\mathbf{r_1}, \ldots, \mathbf{r_A})$.
\\The HF approximation states that the wavefunction can be approximated as a product of single particle states:
\begin{equation}
    \Psi(\mathbf{r_1}, \ldots, \mathbf{r_A}) = \prod_{i=1}^A \phi_i(\mathbf{r_i})
\end{equation}
Since we are dealing with fermions, the correct state must be antisymmetric with respect to a particle exchange, forcing the use of a slater determinant:
\begin{equation}
    \Psi(\mathbf{r_1}, \ldots, \mathbf{r_A}) = \frac{1}{\sqrt{A!}}\det
    \begin{pmatrix}
        \phi_1(\mathbf{r_1}) & \ldots & \phi_A(\mathbf{r_A}) \\
        \ldots & \ddots & \ldots \\
        \phi_1(\mathbf{r_A}) & \ldots & \phi_A(\mathbf{r_1})
    \end{pmatrix}
    =\sum_{p} (-1)^p \phi_{p(1)}(\mathbf{r_1}) \ldots \phi_{p(A)}(\mathbf{r_A})
\end{equation}
Where the sum is performed over all possible permutations of the particles.
\\Using the variational principle, we can determine the ground state by minimizing the energy functional
\begin{equation}
    E[\Psi] = \bra{\Psi} \hat{H} \ket{\Psi}
\end{equation}
With the constraint that the single particle states be orthogonal to each other.
\begin{equation}
    \delta E = \delta(\bra{\Psi} \hat{H} \ket{\Psi} - \sum_A \varepsilon_i\braket{\phi_i | \phi_i} ) = 0
\end{equation}
