\section{The Hartree-Fock method}
A many body system, like the nucleus, is made of indistinguishible particles from the standpoint of quantum mechanics.
\\Suppose to have a nucleus, with $A$ nucleons, where the mass difference between neutrons and protons is neglected.
\\The state of the system will be described by a wavefunction $\Psi(\mathbf{r_1}, \ldots, \mathbf{r_A})$.
\\The HF approximation states that the wavefunction can be approximated as a product of single particle states:
\begin{equation}
    \Psi(\mathbf{r_1}, \ldots, \mathbf{r_A}) = \prod_{i=1}^A \phi_i(\mathbf{r_i})
\end{equation}
Since we are dealing with fermions, the correct state must be antisymmetric with respect to a particle exchange, forcing the use of a slater determinant:
\begin{equation}
    \Psi(\mathbf{r_1}, \ldots, \mathbf{r_A}) = \frac{1}{\sqrt{A!}}\det
    \begin{pmatrix}
        \phi_1(\mathbf{r_1}) & \ldots & \phi_A(\mathbf{r_A}) \\
        \ldots & \ddots & \ldots \\
        \phi_1(\mathbf{r_A}) & \ldots & \phi_A(\mathbf{r_1})
    \end{pmatrix}
    =\sum_{p} (-1)^p \phi_{p(1)}(\mathbf{r_1}) \ldots \phi_{p(A)}(\mathbf{r_A})
\end{equation}
Where the sum is performed over all possible permutations of the particles.
\\Using the variational principle, we can determine the ground state by minimizing the energy functional
\begin{equation}
    \label{eq:energy_functional}
    E[\Psi] = \bra{\Psi} \hat{H} \ket{\Psi}
\end{equation}
With the constraint that the single particle states be orthogonal to each other.
\begin{equation}
    \label{eq:energy_variation}
    \delta E = \delta(\bra{\Psi} \hat{H} \ket{\Psi} - \sum_A \varepsilon_i\braket{\phi_i | \phi_i} ) = 0
\end{equation}
This variation, along with the constraint, gives rise to a single particle Schrodinger-like equation:
\begin{equation}
-\frac{\hbar^2}{2m} \nabla_i^2 \phi_i(\mathbf{r}) + \sum_{j=1}^A \int d^3 r' \ \phi_j^*(\mathbf{r}') v(\mathbf{r}, \mathbf{r}') \phi_j(\mathbf{r}') \phi_i(\mathbf{r}) - \sum_{j=1}^A \int d^3 r' \, \phi_j^*(\mathbf{r}') v(\mathbf{r}, \mathbf{r}') \phi_j(\mathbf{r}) \phi_i(\mathbf{r}') = \varepsilon_i \phi_i(\mathbf{r}).
\end{equation}
One can choose a proper potential $v(\mathbf{r_1}, \mathbf{r_2})$, a trial set of wavefunctions, and solve for $(\phi_i, \varepsilon_i)$.
\\Since the equation changes for the new solutions, one can use an iterative procedure to solve the self consistent problem.
\subsection{Formal treatment}
Often there is the need of expanding the coordinate space representation of the single particle hamiltonian solution on an arbitrary basis.
\begin{equation}
    \phi_k = \sum_i D_{ik} \chi_i
\end{equation}
Defining the corresponding creation/annihilation operators $c_l^\dagger, c_l$ for $\{\chi_l\}$ and $a_l^\dagger, a_l$ for $\{\phi_k\}$, the transformation between the two will be
\begin{equation}
   a_k^\dagger = \sum_l D_{lk} c_l^\dagger
\end{equation}
Since we're dealing with otrhonormal, complete basis sets, it's trivial to show that the transformation matrix $D$ is unitary.
\\This also implies that the c/a operators commute and obey separate anticommutation relations.
