\subsection{Functional}
For $\boldsymbol{s} = 0$ (even-even nuclei), the functional reduces to the following terms
\subsubsection{t0 t3 Interaction}
The $t_0$ part of the functional is given by
\[\mathcal H_0 = \frac 1 2 t_0 [(1+\frac {x_0} 2)\rho^2 - (x_0 + \frac 1 2 )(\rho_p^2 + \rho_n^2)]\]
Taking the variation with respect to $\rho_q$, we get
\begin{align}
    \fdv{H_0}{\rho_q} &= \frac 1 2 t_0 [(2 + x_0)\rho - (2x_0 + 1)\rho_q]
\end{align}
Which in the case where $\rho_n = \rho_p = \rho_q$, $x_0 = 0$ gives
\[\fdv{H_0}{\rho_q} =\frac 1 2  t_0 (2 \rho - \rho_q ) \]
The $t_3$ part of the functional is given by
\[\mathcal H _3 = \frac 1 {24} t_3 \rho^\sigma [(2+x_3)\rho^2 -(2x_3+1)(\rho_p^2 + \rho_n^2)]\]
Taking the variation with respect to $\rho_q$, we get
\begin{align}
    \fdv{H_3}{\rho_q} &= \frac 1 {24} t_3  [(2+x_3)(\sigma + 2)\rho^{\sigma+1} - (2x_3+1)(\sigma \rho^{\sigma-1}(\rho_n^2+\rho_p^2) +2\rho^\sigma \rho_q)]
    \\&= \frac 1 {24} t_3 \rho^\sigma [(2+x_3)(\sigma + 2)\rho - (2x_3+1)(\sigma \rho^{-1}(\rho_n^2+\rho_p^2) +2\rho_q)]
    \\&= \frac 1 {12} t_3 \rho^\sigma [(1+\frac{x_3}{2})(\sigma + 2)\rho - (x_3+\frac 1 2 )(\sigma \rho^{-1}(\rho_n^2+\rho_p^2) +2\rho_q)]
\end{align}
In the case where $\rho_n = \rho_p = \rho_q$, $x_3 = 0,\ \sigma = 1$, we get
\[\fdv{H_3}{\rho_q} = \frac 1 {24} t_3 \rho [6\rho - 3\rho_q] = \frac{t_3}{4}(\rho^2 - \rho_q^2)\] 
Assuming only $t_0$ and $t_3$ to be non zero parameters, we get the following single particle equation
\begin{equation}
    \bigg(-\frac{\hbar^2}{2m} \nabla^2 + \fdv{\mathcal H_0}{\rho_q} + \fdv{\mathcal H_3}{\rho_q}\bigg)\phi_\alpha = \varepsilon_\alpha\phi_\alpha
\end{equation}

\subsection{Full interaction, neglecting Coulomb, S.O. and S.G. terms}
We can add further terms, which will also depend on $\tau_q, \nabla\rho_q$.
\\The part giving rise to an effective mass develops as 
\begin{align}
    \mathcal H_\text{eff} = &+\frac 1 8 [t_1(2+x_1)+t_2(2+x_2)]\rho\tau  \\&+ \frac 1 8 [t_2(1+2x_2)-t_1(1+2x_1)]\rho_q\tau_q
    \\\fdv{\mathcal H_\text{eff}}{\rho_q} =&+ \frac 1 8 [t_1(2+x_1)+t_2(2+x_2)]\tau \\& + \frac 1 8 [t_2(1+2x_2)-t_1(1+2x_1)]\tau_q
    \\\fdv{\mathcal H_\text{eff}}{\tau_q} =&+ \frac 1 8 [t_1(2+x_1)+t_2(2+x_2)]\rho\\ &+ \frac 1 8 [t_2(1+2x_2)-t_1(1+2x_1)]\rho_q
    \end{align}
While the finite range term 
\begin{align}
    \mathcal H_\text{fin} = &+\frac 1 {32} [3t_1(2+x_1)-t_2(2+x_2)] | \nabla\rho|^2 \\&- \frac 1 {32} [3t_1(2x_1+1)+t_2(2x_2+1)] ( |\nabla\rho_p| ^2 + |\nabla\rho_n|^2) 
\end{align}
Having in mind the relation 
\[\mathcal F [\rho] = \int|\nabla\rho|^2d\mathbf r \implies \fdv {\mathcal F} {\rho} = -2\nabla^2 \rho\]
Gives
\begin{align}
    \fdv{\mathcal H_\text{fin}}{\rho_q} = &+ \frac 1 {16} [t_2(2+x_2)-3t_1(2+x_1)] \nabla^2 \rho \\&+ \frac 1 {16} [3t_1(2x_1+1)+t_2(2x_2+1)] ( \nabla^2 \rho_q )
\end{align}
The final single particle equation will now have effective mass terms and further ones in the mean field
\begin{align}
    \bigg(-\nabla \frac {\hbar^2}{2m^*(\mathbf r)} \nabla + \fdv{(\mathcal{H}_0 + \mathcal{H}_3 + \mathcal{H}_\text{eff} + \mathcal{H}_\text{fin})}{\rho_q} \bigg)\phi_\alpha = \varepsilon_\alpha\phi_\alpha
\end{align}
Where 
\[\frac{\hbar^2}{2m^*(\mathbf r)} = \frac{\hbar^2}{2m} + \fdv{\mathcal{H}_\text{eff}}{\tau_q}\]
\subsubsection{Kinetic term}
Regarding the kinetic component, we end up with an effective mass, such that 
\[\frac{\hbar^2}{2m^*(\mathbf r)} = \frac{\hbar^2}{2m} + \frac 1 8 [t_1(2+x_1)+t_2(2+x_2)]\rho(\mathbf r) - \frac 1 8 [t_1(1+2x_1)+t_2(1+2x_2)]\rho_q(\mathbf r ) = \mu(\mathbf r)\]
This allows us to write
\begin{align}
    \nabla\bigg( \frac{\hbar^2}{2m^*(\mathbf r)} \nabla \phi\bigg)&= \nabla \mu(\mathbf r)\cdot \nabla \phi +\mu(\mathbf r)\nabla^2 \phi 
\end{align}
Where both $\nabla\cdot, \nabla^2$ can be readily evaluated in the previously illustrated finite difference scheme as matrix coefficients.
\subsection{Spin Orbit interaction}
The spin orbit interaction arises by including 
\begin{align}
    \mathcal H_\text{so} = \frac 1 2 W_0[\mathbf J \cdot \nabla \rho + \mathbf J_p \cdot \nabla \rho_p + \mathbf J_n \cdot \nabla \rho_n]
\end{align}
Which adds a contribution to the mean field
\begin{equation}
    \fdv{\mathcal H_\text{so}}{\rho_q} = -\frac 1 2 W_0 [\nabla \cdot \mathbf J + \nabla \cdot \mathbf J_q]
\end{equation}
And a contribution to the spin orbit interaction
\begin{equation}
    \mathbf W_q (\mathbf r ) = \frac 1 2 W_0 [\nabla\rho + \nabla \rho_q]
\end{equation}
\subsection{Spin Gradient coupling}
The remainging Skyrme interaction term is given by
\[
\mathcal H_\text{sg} = -\frac 1 {16} (t_1 x_1 + t_2 x_2) \mathbf J ^2 + \frac 1 {16}(t_1 - t_2) [\mathbf J^2_p + \mathbf J^2_n]
\]
Whose variation with respect to $\mathbf J_q$ adds the following contribution to the spin-orbit term
\[
\fdv{\mathcal H_\text{sg}}{\mathbf J_q} = - \frac 1 8 (t_1 x_1 + t_2 x_2) \mathbf J + \frac 1 8 (t_1 - t_2) \mathbf J_q
\]
\subsection{Summary}
The full interaction reads 
\begin{align*}
    \frac{\hbar^2}{2m^*(\mathbf r)} &= \frac{\hbar^2}{2m} + \frac 1 8 [t_1(2+x_1)+t_2(2+x_2)]\rho(\mathbf r) - \frac 1 8 [t_1(1+2x_1)+t_2(1+2x_2)]\rho_q(\mathbf r ) \\\\
    U_q(\mathbf r) &= \frac 1 8 [t_1(2+x_1)+t_2(2+x_2)]\rho + \frac 1 8 [t_2(1+2x_2)-t_1(1+2x_1)]\rho_q \\
    &+ \frac 1 8 [t_1(2+x_1)+t_2(2+x_2)]\tau + \frac 1 8 [t_2(1+2x_2)-t_1(1+2x_1)]\tau_q \\
    &+ \frac 1 {16} [t_2(2+x_2)-3t_1(2+x_1)] \nabla^2 \rho + \frac 1 {16} [3t_1(2x_1+1)+t_2(2x_2+1)] \nabla^2 \rho_q \\\\
    \mathbf W_q (\mathbf r ) &= \frac 1 2 W_0 [\nabla\rho + \nabla \rho_q] -\frac 1 8 (t_1 x_1 + t_2 x_2) \mathbf J + \frac 1 8 (t_1 - t_2) \mathbf J_q 
\end{align*}

