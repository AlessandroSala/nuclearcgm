
\subsection{Skyrme interaction}
In nuclear physics, the interaction between two nucleons has a very complex functional form. This requires choosing an interaction potential which enables us to simplify the equations and solve the problem.
\\One of such potentials is the Skyrme interaction, whose formulation is based on the physical principle of a short range interaction among nucleons, consequence of the exchange of massive bosons as the force mediators.
\\It is made up of two parts, a two body interaction and a three body interaction.
\[V=\sum_{i<j}v_{ij}^{(2)}+\sum_{i<j<k}v_{ijk}^{(3)}\]
The standard modern parametrization for $v^{(2)}$ is (Chabanat 1998)
\begin{align*}
v^{(2)}(\mathbf{r}_1, \mathbf{r}_2) &= t_0 \left(1 + x_0 P_\sigma \right) \delta(\mathbf{r}) \\
&\quad + \frac{1}{2} t_1 \left(1 + x_1 P_\sigma \right) \left[ \mathbf{P}'^2 \delta(\mathbf{r}) + \delta(\mathbf{r}) \mathbf{P}^2 \right] \\
&\quad + t_2 \left(1 + x_2 P_\sigma \right) \mathbf{P}' \cdot \delta(\mathbf{r}) \mathbf{P} \\
&\quad + \frac{1}{6} t_3 \left(1 + x_3 P_\sigma \right) \left[ \rho(\mathbf{R}) \right]^\sigma \delta(\mathbf{r}) \\
&\quad + i W_0 \boldsymbol{\sigma}\cdot \left[ \mathbf{P}' \times \delta(\mathbf{r}) \mathbf{P} \right]
\end{align*}
Where 
\[\mathbf{r} = \mathbf{r}_1 - \mathbf{r}_2\]
\[\mathbf{R} = \frac{\mathbf{r}_1+\mathbf{r}_2}{2}\]
\[\mathbf{P} = -i(\nabla_1 - \nabla_2)/2\]
\[\boldsymbol{\sigma} = \boldsymbol{\sigma}_1 + \boldsymbol{\sigma}_2\]
\[\mathbf{P}_\sigma = (1+\boldsymbol{\sigma}_1\cdot\boldsymbol{\sigma}_2)/2\]
Primed operators refer to the complex conjugate acting on the left.
\\This formulation respects all symmetries required of a non relativistic nuclear interaction (Galilean boost, particle exchange, translation, rotation, parity, time inversion and translation).
\subsubsection{Three body interaction}
The three body term of the Skyrme force is encapsulated by the term 
\[\frac 1 6 t_3 \left(1 + x_3 P_\sigma \right) \left[ \rho(\mathbf{R}) \right]^\sigma \delta(\mathbf{r}) \]
Here, $\sigma$ in the exponent is a free parameter of the force.
\subsubsection{Energy functional}
Evaluating the energy functional \ref{eq:energy_functional}, we get
\begin{align}
    \bra{\Psi} H \ket{\Psi} = \int H(\mathbf{r}) d^3 r 
    \\\mathcal H(\mathbf{r}) = \mathcal{K} + \mathcal{H}_0 + \mathcal{H}_3 + \mathcal{H}_\text{eff}+\mathcal{H}_\text{fin}+\mathcal{H}_\text{so} + \mathcal{H}_\text{sg} + \mathcal{H}_\text{coul}
\end{align}
\subsection{Density matrix and associated quantities}
$\mathcal H$ depends on $\mathbf r$ through a series of known and physically relevant quantities.
\\Starting from the density matrix, defined as
\begin{equation}
    \rho_q (\mathbf r \sigma, \mathbf r \sigma') = \sum_{\alpha} \phi_{\alpha, \sigma} (\mathbf r )\phi_{\alpha, \sigma'}^*(\mathbf r')
\end{equation}
The index $\alpha$ goes through all single particle states of the particles of type $q$ (Protons, Neutrons), while the index $\sigma$ refers to the spin coordinate (Up, Down). 
\begin{align}
    \rho_q(\mathbf r, \mathbf r') &= \sum_{\sigma}\rho(\mathbf r\sigma, \mathbf r'\sigma) =\sum_{\alpha} \phi_{\uparrow}(\mathbf r)\phi_{\uparrow}^*(\mathbf r')+\phi_{\downarrow}(\mathbf r)\phi_{\downarrow}^*(\mathbf r')
    \\\rho_q(\mathbf r) &= \rho_q(\mathbf r, \mathbf r')\bigg|_{\mathbf r'=\mathbf r} =\sum_{\alpha} |\phi_{\uparrow}(\mathbf r)|^2+|\phi_{\downarrow}(\mathbf r)|^2
    \\\tau_q(\mathbf r) &= \sum_{\alpha} \nabla'\cdot\nabla\rho_q(\mathbf r, \mathbf r')\bigg|_{\mathbf r'=\mathbf r} 
    \\&= \sum_{\sigma, \alpha} \nabla \phi_\sigma (\mathbf r)\cdot \nabla \phi_\sigma^*(\mathbf r')\bigg|_{\mathbf r = \mathbf r'} = \sum_{\sigma, \alpha} |\nabla \phi_\sigma(\mathbf r)|^2 
    \\&= \sum_{\alpha}|\nabla \phi_\uparrow(\mathbf r)|^2 + |\nabla \phi_\downarrow(\mathbf r)|^2
    \\s_q(\mathbf r, \mathbf r') &=\sum_{\sigma \sigma', i} \rho_q(\mathbf r \sigma, \mathbf r' \sigma')\bra{\sigma'} \hat {\boldsymbol{\sigma}} \ket{\sigma} = \sum_{\alpha} \begin{bmatrix} \phi_{\uparrow}^*(\mathbf r') \ \phi_{\downarrow}^*(\mathbf r') \end{bmatrix}\hat{\boldsymbol{\sigma}} \begin{bmatrix} \phi_{\uparrow}(\mathbf r) \\ \phi_{\downarrow}(\mathbf r) \end{bmatrix}
    \\J_{q, \mu\nu} &= \frac 1 {2i}(\partial_\mu - \partial_\mu') s_{q, \nu}(\mathbf r, \mathbf r')\bigg|_{\mathbf r'=\mathbf r}\\
    &= \frac 1 {2i}\bigg(\begin{bmatrix}\phi_{\uparrow}^*(\boldsymbol r')\ \phi_{\downarrow}^*(\boldsymbol r')\end{bmatrix} \partial_\mu\hat{\sigma}_\nu\begin{bmatrix} \phi_{\uparrow}(\mathbf r) \\ \phi_{\downarrow}(\mathbf r) \end{bmatrix} - \begin{bmatrix}\phi_{\uparrow}(\boldsymbol{r})\ \phi_{\downarrow}(\boldsymbol{r})\end{bmatrix} \partial_\mu'\hat{\sigma}_\nu\begin{bmatrix} \phi_{\uparrow}^*(\mathbf r') \\ \phi_{\downarrow}^*(\mathbf r') \end{bmatrix}\bigg)_{\mathbf r'=\mathbf r}
     \\&= \sum_\alpha\Im\bigg\{\begin{bmatrix}\phi_{\uparrow}^*(\boldsymbol r)\ \phi_{\downarrow}^*(\boldsymbol r) \end{bmatrix}\partial_\mu \hat{\sigma}_\nu\begin{bmatrix} \phi_{\uparrow}(\mathbf r) \\ \phi_{\downarrow}(\mathbf r) \end{bmatrix}\bigg\}
\end{align}
Where $\alpha$ goes through all single particle states, the index is omitted on $\phi$ for brevity.
\\Taking the variation of $E[\Psi]$ with respect to $\phi_i^*$ we get a single particle equation
\begin{equation}
   \bigg(-\nabla \frac{\hbar^2}{2m^*(\mathbf r)} \nabla + U_q(\mathbf r)+\delta_{q, \text{proton}}V_c(\mathbf r)-  i \mathbf W_q \cdot (\nabla \times \boldsymbol{\sigma)} \bigg)\phi_\alpha = \varepsilon_\alpha\phi_\alpha 
\end{equation}
We will now see how to properly treat each term of the equation.
\subsection{Functional}
For $\boldsymbol{s} = 0$ (even-even nuclei), the functional reduces to the following terms, assuming only $t_0, t_3$ different from zero
\subsubsection{t0}
The $t_0$ part of the functional is given by
\[\mathcal H_0 = \frac 1 2 t_0 [(1+\frac {x_0} 2)\rho^2 - (x_0 + \frac 1 2 )(\rho_p^2 + \rho_n^2)]\]
Taking the variation with respect to $\rho_q$, we get
\begin{align}
    \fdv{H_0}{\rho_q} &= \frac 1 2 t_0 [(2 + x_0)\rho - (2x_0 + 1)\rho_q]
\end{align}
Which in the case where $\rho_n = \rho_p = \rho_q$, $x_0 = 0$ gives
\[\fdv{H_0}{\rho_q} =\frac 1 2  t_0 (2 \rho - \rho_q ) \]
\subsubsection{t3}
The $t_3$ part of the functional is given by
\[\mathcal H _3 = \frac 1 {24} t_3 \rho^\sigma [(2+x_3)\rho^2 -(2x_3+1)(\rho_p^2 + \rho_n^2)]\]
Taking the variation with respect to $\rho_q$, we get
\begin{align}
    \fdv{H_3}{\rho_q} &= \frac 1 {24} t_3  [(2+x_3)(\sigma + 2)\rho^{\sigma+1} - (2x_3+1)(\sigma \rho^{\sigma-1}(\rho_n^2+\rho_p^2) +2\rho^\sigma \rho_q)]
    \\&= \frac 1 {24} t_3 \rho^\sigma [(2+x_3)(\sigma + 2)\rho - (2x_3+1)(\sigma \rho^{-1}(\rho_n^2+\rho_p^2) +2\rho_q)]
    \\&= \frac 1 {12} t_3 \rho^\sigma [(1+\frac{x_3}{2})(\sigma + 2)\rho - (x_3+\frac 1 2 )(\sigma \rho^{-1}(\rho_n^2+\rho_p^2) +2\rho_q)]
\end{align}
In the case where $\rho_n = \rho_p = \rho_q$, $x_3 = 0,\ \sigma = 1$, we get
\[\fdv{H_3}{\rho_q} = \frac 1 {24} t_3 \rho [6\rho - 3\rho_q] = \frac{t_3}{4}(\rho^2 - \rho_q^2)\] 



%\subsection{Total energy calculation}
\newpage
\subsection{Coulomb interaction}
In the Slater approximation, the Coulomb interaction energy contribution to the energy density reads
\begin{equation}
    \mathcal H_\text{coul} = \frac{e^2}{2}\bigg[\iint  \frac{\rho_p(\mathbf r )\rho(\mathbf r ' )}{|\mathbf r-\mathbf r'|}d^3\mathbf r d^3\mathbf r' - \frac 3 2 \bigg(\frac 3 \pi \bigg) ^{\frac 1 3}\int \rho_p^{4/3}(\mathbf r)d^3\mathbf r\bigg]
\end{equation}
Which gives
\begin{equation}
    V_{c, D+E}(\mathbf r) = \frac{e^2}{2}\bigg[\int \frac{\rho_p(\mathbf r ')}{|\mathbf r-\mathbf r'|} d^3 \mathbf r' - 2\bigg(\frac 3 \pi \bigg) ^{\frac 1 3} \rho_p^{1/3}(\mathbf r ) \bigg]
\end{equation}
From a computational standpoint, the exchange part is trivial, while the direct one is more involved.
\\One could compute the integral, but the complexity is $\mathcal O(N^6)$, rendering it impractical for 3d meshes.
\\An alternative approach is to solve the poisson equation (from now on, $V_c$ refers to the direct part only)
\begin{equation}
    \nabla^2 V_c = 4\pi e^2 \rho_p
\end{equation}
On the considered mesh, using the already defined finite differences approach.
\\The boundary conditions are of Dirichlet type, which can be extracted from a quadrupole expansion of the charge density
\begin{equation}
V_c (\mathbf r) = 4\pi e^2 \sum_{\lambda=0}^2\sum_{\mu=-\lambda}^\lambda \frac{\expval{Q_{\lambda\mu}} Y_{\lambda\mu}}{r^{1+\lambda}}\text{ on }\partial V
\end{equation}
Where $\expval{Q_{\lambda\mu}}$ is defined as 
\begin{equation}
    \expval{Q_{\lambda\mu}} = \int r^\lambda Y_{\lambda\mu}^* (\mathbf r)\rho_p(\mathbf r ) d^3 \mathbf r
\end{equation}
Since we expect a charge density confined to the nuclear shape, higher order terms in the expansion can be neglected, provided that the box is sufficiently large.
\\In a reference frame where the nucleus center of mass is at the origin, the expansion reduces to
\begin{equation}
    V_{c}(\mathbf r ) = \frac{Ze^2}{r} + e^2\sum_{\mu=-2}^{2}\frac{\expval{Q_{2\mu}}Y_{2\mu}}{r^3} \text{ on } \partial \Omega
\end{equation}
\subsubsection{Imposing the boundary conditions}
The high order discretization stencil of the laplacian, like in our case, involves points with a distance up to $\pm 2$, this means that the quadrupole expanded potential must be calculated on a box bigger than the one used.
\\Without losing generality, we can look at how boundary conditions are set in 1D. The 3D case follows trivially.
\\The linear system of equations will look like
\[
\begin{bmatrix}
    c_{0,0} & \ldots & c_{0,N-1} \\
    \vdots & \ddots & \vdots \\
    c_{N-1,0} & \ldots & c_{N-1,N-1}
\end{bmatrix}
\begin{bmatrix}
    V_0 \\ \vdots \\ V_{N-1}
\end{bmatrix}
=
4\pi e^2
\begin{bmatrix}
    \rho_0 \\ \vdots \\ \rho_{N-1}
\end{bmatrix}
\]
Near a boundary, say $i=0$, the stencil has the form
\begin{equation}
    c_{0,-2}V_{-2} + c_{0,-1}V_{0,-1} + c_{0,0}V_0 + c_{0,1}V_1 + c_{0,2}V_2 = 4\pi e^2 \rho_0
\end{equation}
The $C$ matrix only accounts for points inside the box, namely $i=0, 1, 2$ we can take advantage of this and bring the "missing" terms on the right side of the equation, calculated from the multipole expansion.
\[
c_{0,0} V_0 + c_{0,1} V_1 + c_{0,2} V_2 = 4\pi e^2 \rho_0 - c_{0,-2}V_{-2} - c_{0,-1}V_{-1}
\]
\\This forces the linear system to always abide by the boundary conditions.
\\This same procedure must be applied to every equation involving points outside the box, e.g. for $i=1$
\[
c_{1,0} V_0 + c_{1,1} V_1 + c_{1,2} V_2 + c_{1,3} V_3 = 4\pi e^2 \rho_1 - c_{1,-1}V_{-1} 
\]
Finally, we can compute the potential vector using the simple conjugate gradient.

\section{Spherical harmonics}
Spherical harmonics, of order $\lambda, \mu$, are defined as
\begin{equation}
    Y_{\lambda\mu} (\theta, \phi) = (-1)^\mu 
    \sqrt{\frac{2\lambda + 1}{4\pi}\frac{(\lambda-\mu)!}{(\lambda+\mu)!}}
    \, P_\lambda^\mu(\cos\theta) \, e^{i\mu\phi}.
\end{equation}
Being able to provide the expression for arbitrary $\mu, \lambda$ through an algorithm 
is important in the current framework, to solve the Poisson equation and investigate 
nuclear properties.
\\The major challenge is to generate the associated Legendre polynomials $P_\lambda^\mu$.
They can be expressed in the form (for positive $\mu$)
\begin{equation}
    P_\lambda^\mu(x) = (1-x^2)^{\mu/2} \dv[\mu]{P_\lambda(x)}{x},
\end{equation}
where $x = \cos\theta$ and
\begin{equation}
    P_\lambda(x) = \frac 1 {2^\lambda \lambda !}
    \dv[\lambda]{(x^2-1)^\lambda}{x}.
\end{equation}
To compute the arbitrary $\lambda, \mu$ associated Legendre polynomial we can employ a recursive approach, setting $\lambda =\mu$
\begin{equation}
    P_\mu^\mu(x) = (2\mu-1)!! \, (1-x^2)^{\mu/2},
\end{equation}
where $(2\mu-1)!! = 1\cdot 3 \cdot 5 \ldots (2\mu-1)$ denotes the double factorial.
Once $P_\mu^\mu(x)$ is known, the next element with $\lambda = \mu +1$ reads
\begin{equation}
    P_{\mu+1}^\mu(x) = x(2\mu+1)P_\mu^\mu(x).
\end{equation}
All higher orders are then generated using the standard upward recurrence relation in $\lambda$:
\begin{equation}
    (\lambda - \mu + 1) \, P_{\lambda+1}^\mu(x) =
    (2\lambda + 1) \, x \, P_\lambda^\mu(x) -
    (\lambda + \mu) \, P_{\lambda-1}^\mu(x),
\end{equation}
valid for all $\lambda \geq \mu+1$.  
\paragraph{Summarized algorithm}
\begin{enumerate}
    \item Compute the base case $P_\mu^\mu$ from the closed-form formula.
    \item If $\mu = \lambda$ the procedure ends, otherwise
    \item Evaluate $P_{\mu+1}^\mu$, if $\lambda = \mu +1$ the procedure ends, otherwise
    \item Apply the recurrence relation $P_{\lambda+1}^\mu$ until the desired degree is reached
\end{enumerate}
This ought to be applied only for $ \mu \ge 0$. For $\mu < 0$ the procedure is carried out using $-\mu$ and in the end using the relation
\begin{equation}
    Y_{\lambda-\mu} = (-1)^{\mu}Y_{\lambda\mu}^{*}
\end{equation}

