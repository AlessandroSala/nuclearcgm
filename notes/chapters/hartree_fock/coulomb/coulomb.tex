\subsection{Coulomb interaction}
In the usual Slater approximation, the Coulomb interaction energy contribution to the energy density reads
\begin{equation}
    \mathcal H_\text{coul} = \frac{e^2}{2}\bigg[\iint  \frac{\rho_p(\mathbf r )\rho(\mathbf r ' )}{|\mathbf r-\mathbf r'|}d^3\mathbf r d^3\mathbf r' - \frac 3 2 \bigg(\frac 3 \pi \bigg) ^{\frac 1 3}\int \rho_p^{4/3}(\mathbf r)d^3\mathbf r\bigg]
\end{equation}
Which gives
\begin{equation}
    V_c(\mathbf r) = \frac{e^2}{2}\bigg[\int \frac{\rho_p(\mathbf r ')}{|\mathbf r-\mathbf r'|} d^3 r' - 2\bigg(\frac 3 \pi \bigg) ^{\frac 1 3} \rho_p^{1/3}(\mathbf r ) \bigg]
\end{equation}
From a computational standpoint, the exchange part is trivial, while the direct one is more involved.
\\One could compute the integral, but the complexity is $\mathcal O(N^6)$, rendering it impractical even for coarse meshes.
\\An alternative solution is to solve the poisson equation
\begin{equation}
    \nabla^2 V_c = 4\pi e^2 \rho_p
\end{equation}
Extracting boundary conditions from a quadrupole expansion of the charge density
\begin{equation}
    V_{c, Q}(\mathbf r ) = \frac{Ze^2}{r} + e^2\frac{\expval{Q_{20}} Y_{20}(\mathbf r) + \expval{Q_{22}}\Re Y_{22}(\mathbf r)}{{r}^3}
\end{equation}
