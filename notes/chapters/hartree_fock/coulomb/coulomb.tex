\newpage
\subsection{Coulomb interaction}
In the Slater approximation, the Coulomb interaction energy contribution to the energy density reads
\begin{equation}
    \mathcal H_\text{coul} = \frac{e^2}{2}\bigg[\iint  \frac{\rho_p(\mathbf r )\rho(\mathbf r ' )}{|\mathbf r-\mathbf r'|}d^3\mathbf r d^3\mathbf r' - \frac 3 2 \bigg(\frac 3 \pi \bigg) ^{\frac 1 3}\int \rho_p^{4/3}(\mathbf r)d^3\mathbf r\bigg]
\end{equation}
Which gives
\begin{equation}
    V_{c, D+E}(\mathbf r) = \frac{e^2}{2}\bigg[\int \frac{\rho_p(\mathbf r ')}{|\mathbf r-\mathbf r'|} d^3 \mathbf r' - 2\bigg(\frac 3 \pi \bigg) ^{\frac 1 3} \rho_p^{1/3}(\mathbf r ) \bigg]
\end{equation}
From a computational standpoint, the exchange part is trivial, while the direct one is more involved.
\\One could compute the integral, but the complexity is $\mathcal O(N^6)$, rendering it impractical for 3d meshes.
\\An alternative approach is to solve the poisson equation (from now on, $V_c$ refers to the direct part only)
\begin{equation}
    \nabla^2 V_c = 4\pi e^2 \rho_p
\end{equation}
On the considered mesh, using the already defined finite differences approach.
\\The boundary conditions are of Dirichlet type, which can be extracted from a quadrupole expansion of the charge density
\begin{equation}
V_c (\mathbf r) = 4\pi e^2 \sum_{\lambda=0}^2\sum_{\mu=-\lambda}^\lambda \frac{\expval{Q_{\lambda\mu}} Y_{\lambda\mu}}{r^{1+\lambda}}\text{ on }\partial V
\end{equation}
Where $\expval{Q_{\lambda\mu}}$ is defined as 
\begin{equation}
    \expval{Q_{\lambda\mu}} = \int r^\lambda Y_{\lambda\mu}^* (\mathbf r)\rho_p(\mathbf r ) d^3 \mathbf r
\end{equation}
Since we expect a charge density confined to the nuclear shape, higher order terms in the expansion can be neglected, provided that the box is sufficiently large.
\\In the frequent case of a nucleus in a parity eigenstate, the charge density is even, leading to 
\begin{equation}
    V_{c}(\mathbf r ) = \frac{Ze^2}{r} + e^2\frac{\expval{Q_{20}} Y_{20}(\mathbf r) + \expval{Q_{22}}\Re Y_{22}(\mathbf r)}{{r}^3} \text{ on } \partial V
\end{equation}
\subsubsection{Imposing the boundary conditions}
The high order discretization stencil of the laplacian, like in our case, involves points with a distance up to $\pm 2$, this means that the quadrupole expanded potential must be calculated on a box bigger than the one used.
\\Without losing generality, we can look at how boundary conditions are set in 1D. The 3D case follows trivially.
\\The linear system of equations will look like
\[
\begin{bmatrix}
    c_{0,0} & \ldots & c_{0,N-1} \\
    \vdots & \ddots & \vdots \\
    c_{N-1,0} & \ldots & c_{N-1,N-1}
\end{bmatrix}
\begin{bmatrix}
    V_0 \\ \vdots \\ V_{N-1}
\end{bmatrix}
=
4\pi e^2
\begin{bmatrix}
    \rho_0 \\ \vdots \\ \rho_{N-1}
\end{bmatrix}
\]
Near a boundary, say $i=0$, the stencil has the form
\begin{equation}
    c_{0,-2}V_{-2} + c_{0,-1}V_{0,-1} + c_{0,0}V_0 + c_{0,1}V_1 + c_{0,2}V_2 = 4\pi e^2 \rho_0
\end{equation}
The $C$ matrix only accounts for points inside the box, namely $i=0, 1, 2$ we can take advantage of this and bring the "missing" terms on the right side of the equation, calculated from the multipole expansion.
\[
c_{0,0} V_0 + c_{0,1} V_1 + c_{0,2} V_2 = 4\pi e^2 \rho_0 - c_{0,-2}V_{-2} - c_{0,-1}V_{-1}
\]
\\This forces the linear system to always abide by the boundary conditions.
\\This same procedure must be applied to every equation involving points outside the box, e.g. for $i=1$
\[
c_{1,0} V_0 + c_{1,1} V_1 + c_{1,2} V_2 + c_{1,3} V_3 = 4\pi e^2 \rho_1 - c_{1,-1}V_{-1} 
\]
Finally, we can compute the potential vector using the simple conjugate gradient.
