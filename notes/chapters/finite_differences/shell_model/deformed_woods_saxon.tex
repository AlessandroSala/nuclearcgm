\section{Deformed Woods Saxon}
We can now introduce deformations in the nuclear structure by modifying the effective Woods Saxon potential.
\\More specifically, we can introduce a functional dependence of the nuclear surface $R$ on the direction of the position vector.
\\In the case of an axially symmetric, quadrupole deformation, $R$ will be given by
\begin{equation}
    R(\theta) = R_0 [1+\beta_2 Y_{20}(\theta)]
\end{equation}
Where $Y_{20}(\theta)$ is the spherical harmonic $l=0, \ m=2$, and $\beta_2$ characterizes the degree of deviation from sphericity, and $\theta$ is the angle between the z-axis and the position vector.
\[ Y_{20}(\theta) = \sqrt{\frac{5}{16\pi}}(3\cos^2(\theta)-1)\ \ \   \cos\theta = \frac{z}{\sqrt{x^2+y^2+z^2}}\]
\begin{itemize}
    \item $\beta_2 > 0$ characterizes a prolate nucleus.
    \item $\beta_2 < 0$ characterizes an oblate nucleus.
    \item $\beta_2 = 0$ reduces to the previous spherical case.
\end{itemize}
And the corresponding Woods Saxon potential is given by
\begin{equation}
    V_{\text{WS}}(r) = \frac{-V_0}{1+\exp\bigg(\frac{r-R(\theta)}{a}\bigg)}
\end{equation}
On top of it, we need to adjust the spin orbit potential.
\\In the deformed case, the general form should be used
\begin{equation}
    V_\text{SO}(r) = V_\text{SO} \nabla V_{\text{WS}}\cdot (\bm{\sigma}\times \bm{p})
\end{equation}
Where 
\begin{equation}
    \nabla V_{\text{WS}} = \frac{V_0/a}{\bigg[1+\exp\bigg(\frac{r-R(\theta)}{a}\bigg)\bigg]^2}\exp\bigg(\frac{r-R(\theta)}{a}\bigg)(\nabla r - \nabla R)
\end{equation}
\begin{equation}
    \nabla r = \begin{pmatrix}
               x/r \\ y/r \\ z/r
    \end{pmatrix}
    \ \ \nabla R = R_0\beta_2\dv{Y_{20}}{\theta}\nabla \theta
\end{equation}
\begin{equation}
    \nabla \theta =\frac{1}{r^2\sin\theta} \begin{pmatrix}
               \frac{-xz}{\sqrt{x^2+y^2+z^2}} \\ \frac{-yz}{\sqrt{x^2+y^2+z^2}} \\ \frac{x^2+y^2}{\sqrt{x^2+y^2+z^2}}
    \end{pmatrix}\ \ \dv{Y_{20}}{\theta} = -6\sqrt{\frac{5}{16\pi}}\sin\theta\cos\theta
\end{equation}


